%!TEX program = xelatex
\documentclass{ctexart}
\usepackage{listings}
\usepackage{xcolor}
\usepackage{aligned-overset}
\usepackage{bookmark}
\usepackage{fancyhdr}
\usepackage{geometry}
\usepackage{bm}
\usepackage{chapterbib}
\geometry{a4paper,scale=0.8}
\pagestyle{fancy}
\fancyhf{}
\fancyhead[C]{ACM模板 by 嘤嘤嘤} %页眉与页脚
\fancyfoot[C]{} 
\rfoot{\thepage} % 显示页码
\hypersetup{hidelinks} % 隐藏目录红色边框
\ctexset{section/format = {\Large\bfseries}} % section标题靠左
%\lstset{ % 代码格式
%columns=fixed,       
%    % numbers=left,                                        % 在左侧显示行号
%    frame=none,                                          % 不显示背景边框
%    backgroundcolor=\color[RGB]{245,245,244},            % 设定背景颜色
%    keywordstyle=\color[RGB]{40,40,255},                 % 设定关键字颜色
%    numberstyle=\footnotesize\color{darkgray},           % 设定行号格式
%    commentstyle=\it\color[RGB]{0,96,96},                % 设置代码注释的格式
%    stringstyle=\rmfamily\slshape\color[RGB]{128,0,0},   % 设置字符串格式
%    showstringspaces=false,                              % 不显示字符串中的空格
%    language=c++, 
%   % basicstyle=\footnotesize,
%	basicstyle=\small,  
%    breaklines=true,
%    breakatwhitespace=false
%    }




\definecolor{commentgreen}{RGB}{2,112,10}
\definecolor{eminence}{RGB}{108,48,130}
\definecolor{weborange}{RGB}{255,165,0}
\definecolor{frenchplum}{RGB}{129,20,83}


\lstset {
	language=C++,
	frame=tb,
	tabsize=4,
	showstringspaces=false,
	numbers=left,
	%upquote=true,
	commentstyle=\color{commentgreen},
	keywordstyle=\color{eminence},
	stringstyle=\color{red},
	basicstyle=\small\ttfamily, % basic font setting
	emph={int,char,double,float,unsigned,void,bool},
	emphstyle={\color{blue}},
	escapechar=\&,
	% keyword highlighting
	classoffset=1, % starting new class
	otherkeywords={>,<,.,;,-,!,=,~},
	morekeywords={>,<,.,;,-,!,=,~},
	keywordstyle=\color{weborange},
	classoffset=0,
}












\setmainfont{Courier Regular} % 英文字体
% \setmainfont{Noto Mono Bold} % 英文字体


\begin{document}  

\begin{titlepage} % 封面放在titlepage中不计页码
% \author{嘤嘤嘤}
\title{ACM模板}
\maketitle
\setcounter{page}{0}
\thispagestyle{empty}
\end{titlepage}


\tableofcontents % 显示目录
\newpage


\section{字符串处理}
\subsection{AC自动机}
(by qi)
\lstinputlisting[language=C++]{./src/StringAlgorithm/ACautomation.cpp}
(by shui)
\lstinputlisting[language=C++]{./src/StringAlgorithm/shui/O(n)-AC自动机.cpp}
\subsection{z-algorithm}
\lstinputlisting[language=C++]{./src/StringAlgorithm/z-algorithm.cpp}
\subsection{后缀数组}
(by shui)
\lstinputlisting[language=C++]{./src/StringAlgorithm/shui/O(nlogn)-后缀数组.cpp}
(by qi)
\lstinputlisting[language=C++]{./src/StringAlgorithm/SuffixArray.cpp}
\subsection{最长上升子序列}
\lstinputlisting{./src/StringAlgorithm/shui/O(nlogn)-最长上升子序列.cpp}
\subsection{Manacher}
\lstinputlisting{./src/StringAlgorithm/shui/O(n)-Manacher.cpp}
\subsection{KMP}
\lstinputlisting{./src/StringAlgorithm/shui/O(n+m)-KMP.cpp}
\subsection{ex-KMP}
\lstinputlisting{./src/StringAlgorithm/shui/O(n+m)-扩展KMP.cpp}
\subsection{Sunday}
\lstinputlisting{./src/StringAlgorithm/shui/O(n)-Sunday.cpp}
\subsection{字符串哈希}
\lstinputlisting{./src/StringAlgorithm/shui/O(n)-字符串Hash.cpp}
\subsection{字符串最大最小表示}
\lstinputlisting{./src/StringAlgorithm/shui/O(n)-最大最小表示.cpp}



\section{排序算法}
\subsection{归并排序}
\lstinputlisting{./src/排序算法/归并排序.cpp}



\section{数学}
\subsection{扩展欧几里得算法}
\lstinputlisting{./src/math/exgcd.cpp}
\subsection{求逆元}
\lstinputlisting{./src/math/inv_element.cpp}
\subsection{Miller robin素数检验}
\lstinputlisting{./src/math/Miller_Rabin.cpp}
\subsection{快速傅里叶变换}
\lstinputlisting{./src/math/FFT.cpp}
\subsection{快速数论变换}
\lstinputlisting{./src/math/NTT.cpp}
\subsection{求原根}
\lstinputlisting{./src/math/Primitive_root.cpp}
\subsection{BM黑盒线代}
\lstinputlisting{./src/math/LinearRecurrence.cpp}
\subsection{FWT}
\lstinputlisting{./src/math/FWT.cpp}
\subsection{Simpson积分}
\lstinputlisting{./src/math/simpson_integral.cpp}
\subsection{扩展欧拉定理}
\begin{equation}
a^b\equiv \left\{
\begin{aligned}
& a^{b\%\phi(p)}&  gcd(a,p)=1\\
& a^b           &  gcd(a,p)\neq1,b<\phi(p)\\
& a^{b\%\phi(p)+\phi(p)} & gcd(a,p)\neq1,b\geq\phi(p)
\end{aligned}
\right.
\end{equation}

\subsection{杜教筛}
\begin{align*}
	(f*g)(n)&=\sum_{d|n}f(d)g(n/d) \\
	\sum_{i=1}^n(f*g)(i) &= \sum_{i=1}^n\sum_{d|i}f(d)g(i/d) \\
	&= \sum_{d=1}^ng(d)\sum_{i=1}^{\lfloor n/d\rfloor}f(i) \\
	&= \sum_{d=1}^ng(d)s(\lfloor n/d \rfloor) \\
	&= \sum_{d=2}^ng(d)s(\lfloor n/d \rfloor)+g(1)s(n) \\
	g(1)s(n) &=\sum_{i=1}^n(f*g)(i)-\sum_{d=2}^ng(d)s(n/d)
\end{align*}
\lstinputlisting{./src/math/djs.cpp}


\subsection{莫比乌斯反演}
$$ f(n)=\sum_{d|n}g(d) \Longleftrightarrow g(n)=\sum_{d|n}\mu(d)f(\frac{n}{d}) $$


\subsection{gcd}
\lstinputlisting{./src/math/数论/GCD.cpp}
\subsection{Eratosthenes素数筛}
\lstinputlisting{./src/math/数论/Eratosthenes素数筛.cpp}
\subsection{扩展BSGS}
\lstinputlisting{./src/math/数论/扩展BSGS.cpp}
\subsection{欧拉函数}
\lstinputlisting{./src/math/数论/欧拉函数.cpp}
\subsection{矩阵快速幂}
\lstinputlisting{./src/math/数论/矩阵快速幂.cpp}
\subsection{组合数}
\lstinputlisting{./src/math/数论/组合数.cpp}
\subsection{长整型无脑取模乘}
\lstinputlisting{./src/math/数论/长整型无脑取模乘.cpp}
% section  (end)

%%%------------------------------------------------------------




\section{数据结构}
\subsection{左偏树\&优先队列} % (fold)
\lstinputlisting{./src/DataStructures/树/左偏树_优先队列.cpp}
\subsection{树链剖分}
(by shui)
\lstinputlisting{./src/DataStructures/树/树链剖分.cpp}
(by qi)
\lstinputlisting{./src/DataStructures/HeavyLightdeComposition.cpp}
\subsection{Treap}
\lstinputlisting{./src/DataStructures/树/二叉树/Treap.txt}
\lstinputlisting{./src/DataStructures/树/二叉树/Treap指针.cpp}
\lstinputlisting{./src/DataStructures/树/二叉树/Treap数组.cpp}
\subsection{线段树}
\lstinputlisting{./src/DataStructures/树/二叉树/zkw线段树.cpp}
\subsection{主席树}
(by shui)
\lstinputlisting{./src/DataStructures/树/二叉树/主席树.cpp}
(by qi)
\lstinputlisting{./src/DataStructures/PersistentSegmentTree.cpp}
\subsection{二叉查找树}
\lstinputlisting{./src/DataStructures/树/二叉树/二叉查找树_temp.cpp}
\subsection{Splay}
\lstinputlisting{./src/DataStructures/树/二叉树/伸展树(splay_tree).cpp}
\lstinputlisting{./src/DataStructures/树/二叉树/伸展树splay数组.cpp}
\subsection{线段树}
\lstinputlisting{./src/DataStructures/树/二叉树/线段树.cpp}
\subsection{二叉堆\&优先队列}
\lstinputlisting{./src/DataStructures/树/堆/二叉堆_优先队列.cpp}
\subsection{树状数组} % (fold)
\lstinputlisting{./src/DataStructures/树状数组/O(logn)-树状数组.cpp}
\subsection{二维树状数组} % (fold)
\lstinputlisting{./src/DataStructures/树状数组/二维树状数组.cpp}
\subsection{可持久化Trie树} % (fold)
\lstinputlisting{./src/DataStructures/PersistentTrieTree.cpp}
\subsection{并查集} % (fold)
\lstinputlisting{./src/DataStructures/集合/并查集.cpp}
\subsection{栈\&队列}
\subsubsection{单调栈}
\lstinputlisting{./src/DataStructures/单调栈.cpp}
\subsubsection{队列}
\lstinputlisting{./src/DataStructures/queue.cpp}
\subsubsection{单调队列}
\lstinputlisting{./src/DataStructures/单调队列.cpp}


\section{动态规划}
\subsection{数位dp}
\lstinputlisting{./src/DynamicPrograming/数位DP.cpp}
\subsection{状态压缩}
\lstinputlisting{./src/DynamicPrograming/状态压缩.cpp}
\subsection{背包问题}
\subsubsection{01背包}
\lstinputlisting{./src/DynamicPrograming/背包问题/01背包.cpp}
\subsubsection{多重背包}
\lstinputlisting{./src/DynamicPrograming/背包问题/多重背包.cpp}
\subsubsection{完全背包}
\lstinputlisting{./src/DynamicPrograming/背包问题/完全背包.cpp}







\section{图论}
\subsection{二分图}
\subsubsection{KM}
\lstinputlisting{./src/GraphAlgorithm/二分图/O(n^3)-KM.cpp}
\subsubsection{匈牙利}
\lstinputlisting{./src/GraphAlgorithm/二分图/O(VE)-最大匹配.cpp}
\subsection{拓扑排序}
\subsubsection{DFS拓扑排序}
\lstinputlisting{./src/GraphAlgorithm/拓扑排序/O(V+E)-DFS拓扑排序.cpp}
\subsubsection{Kahn拓扑排序}
\lstinputlisting{./src/GraphAlgorithm/拓扑排序/O(V+E)-Kahn拓扑排序.cpp}
\subsection{最短路}
\subsubsection{堆优化Dijkstra}
\lstinputlisting{./src/GraphAlgorithm/最短路/Dijkstra+Priority_queue.cpp}
\subsubsection{SPFA}
\lstinputlisting{./src/GraphAlgorithm/最短路/O(VE)-SPFA.cpp}
\subsubsection{次短路Dijkstra}
\lstinputlisting{./src/GraphAlgorithm/最短路/次短路-Dijkstra.cpp}
\subsubsection{AStar}
\lstinputlisting{./src/GraphAlgorithm/最短路/AStar.cpp}
\subsection{网络流}
\subsubsection{Dinic}
\lstinputlisting{./src/GraphAlgorithm/网络流/O(V^2E)-Dinic.cpp}
\subsubsection{MCMF}
\lstinputlisting{./src/GraphAlgorithm/网络流/MCMF.cpp}
\subsection{连通图}
\subsubsection{割边}
\lstinputlisting{./src/GraphAlgorithm/连通图/割边_Temp.cpp}
\subsubsection{连通图Tarjan}
\lstinputlisting{./src/GraphAlgorithm/连通图/连通图_Tarjan.cpp}



\section{树}
\subsection{LCA}
\subsubsection{RMQ-ST}
\lstinputlisting{./src/树/LCA-RMQ_ST.cpp}
\subsubsection{Tarjan并查集}
\lstinputlisting{./src/树/LCA_Tarjan_并查集.cpp}
\subsubsection{倍增算法}
\lstinputlisting{./src/树/O(nlogn)-LCA倍增.cpp}
\subsection{最小生成树}
\subsubsection{O(elog2v)-primMST}
\lstinputlisting{./src/树/O(elog2v)-primMST.cpp}
\subsubsection{O(elogv)-prim+heap MST}
\lstinputlisting{./src/树/O(elogv)prim+heapMST.cpp}





\section{计算几何}
\subsection{基础定义}
\lstinputlisting{./src/计算几何/基础定义.cpp}
\subsection{点}
\lstinputlisting{./src/计算几何/点.cpp}
\subsection{线}
\lstinputlisting{./src/计算几何/线.cpp}
\subsection{圆}
\lstinputlisting{./src/计算几何/圆.cpp}
\subsection{三角形}
\lstinputlisting{./src/计算几何/三角形.cpp}
\subsection{凸包}
\lstinputlisting{./src/计算几何/凸包.cpp}


%% start of shui's templates
\subsection{O(n)-求凸包 \& 旋转卡壳}
\lstinputlisting{./src/计算几何/byshui/O(n)-求凸包_旋转卡壳.cpp}
\subsection{两圆相交面积}
\lstinputlisting{./src/计算几何/byshui/两圆相交面积.cpp}
\subsection{两直线交点}
\lstinputlisting{./src/计算几何/byshui/两直线交点.cpp}
\subsection{点在直线上的垂点}
\lstinputlisting{./src/计算几何/byshui/点在直线上的垂点.cpp}
\subsection{矩形面积交}
\lstinputlisting{./src/计算几何/byshui/矩形面积交.cpp}
\subsection{线段类}
\lstinputlisting{./src/计算几何/byshui/线段类.cpp}
\subsection{计算三角形外心}
\lstinputlisting{./src/计算几何/byshui/计算三角形外心.cpp}
%% end of shui's


\section{STL}
\subsection{accumulate}
\lstinputlisting{./src/STL/accumulate.cpp}



\section{其他}
\subsection{RMQ-ST}
\lstinputlisting{./src/其他/RMQ-ST.cpp}
\subsection{一些理论}
\lstinputlisting{./src/其他/一些理论.cpp}
\subsection{随机数和文件输出}
\lstinputlisting{./src/其他/随机数和文件输出.cpp}
\subsection{随机遍历数组}
\lstinputlisting{./src/其他/随机遍历数组.cpp}
\subsection{尺取}
\lstinputlisting{./src/其他/尺取.cpp}

\subsection{std::unordered\_map避免TLE}
\begin{lstlisting}
unordered_map<int,int>mp;
mp.reserve(1024);
mp.max_load_factor(0.25);
\end{lstlisting}
\subsection{输入输出挂}
(by shui)
\lstinputlisting{./src/其他/输入输出优化.cpp}
(by qi)
\lstinputlisting{./src/其他/FastIO.cpp}
\subsection{vim配置}
\lstinputlisting{./src/其他/_vimrc}
\subsection{程序对拍器}
\lstinputlisting[language=bash]{./src/其他/checker.sh}




中文 english





\end{document}
